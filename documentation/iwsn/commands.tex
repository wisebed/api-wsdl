%------------------------------------------------------------------------------
%- PAKETE
%------------------------------------------------------------------------------

		\usepackage{longtable}

		\usepackage{floatrow}

	%Grafiken
		\usepackage{float} %Float-Handling mit Schalter H (gleiche Position wie im Skript)
		\usepackage{placeins} %Barriere f�r Float-Umgebungen erzeugen mit \FloatBarrier
	
	%Verbessertes Beschriften mir div. Optionen
		\usepackage{caption}
	
	%Zusaetzliche Symbole und Schriften (ams: american mathematical soc)
		\usepackage{amssymb}

	%Text Companion fonts which provide many text symbols (such as baht, bullet, copyright, musicalnote, onequarter, section, and yen) in the TS1 encoding.
		\usepackage{textcomp}
	
	%Farbunterst�tzung (ausserhalb der Bilder)
		\usepackage{color}
	
	%Ensure minimal spacing of table cells (http://www.ctan.org/tex-archive/help/Catalogue/entries/cellspace.html)
		\usepackage{cellspace}
	
	%Subfigures
		\usepackage{subfigure}
		
	% package to customize the three basic lists (enumerate, itemize and description) 
	% by means of a set of parameters, and to clone them to define new "logical" lists.
		\usepackage{enumitem}
		\setitemize{enumsep=-3pt}
		\setitemize{itemsep=-3pt}

	%Definitionen
		\usepackage{theorem}
		\newcounter{theorem}
		\newtheorem{definition}[theorem]{Definition}

	%Zitate
		\newcounter{quotectr}
		\newtheorem{myquote}[quotectr]{Quote}

%------------------------------------------------------------------------------
%- Layout
%------------------------------------------------------------------------------

	%Tiefe des Inhaltsverzeichnisses und der Nummerierung der Kapitel
		\setcounter{secnumdepth}{3}
		\setcounter{tocdepth}{3}

	% Listings schoen machen 
		\renewcommand*\ttdefault{txtt}
	
	
	% Absatzformatierungen:
	% Keeps the distance between paragraphs constant
		\setlength{\parskip}{2ex plus 1.0ex minus 1.0ex}
		\setlength{\parindent}{0pt}
	
	% Modify the placement of figures: from faq source: You can adjust the cut-off value if you like, 
	% but it makes no sense to go higher than .95 (LaTeX's default value is only .5). Also, the first 
	% 3 values should be equal, and the last should be 1 - \floatpagefraction.  Otherwise, you are 
	% likely to get floats flushed to the end. 
		\renewcommand{\floatpagefraction}{0.9}
		\renewcommand{\topfraction}{0.9}
		\renewcommand{\bottomfraction}{0.9}
		\renewcommand{\textfraction}{0.1}
		\renewcommand{\textfloatsep}{5mm}
	

%------------------------------------------------------------------------------
%- Textbausteine
%------------------------------------------------------------------------------

	\newcommand{\note}[2][]{{\bf (\ifthenelse{\equal{#1}{}}{}{[#1]: }\marginpar{{\bf\Large [!!!]}}#2})}
	\newcommand{\question}[1]{{\textcolor{red}{\em #1}\marginpar{{\bf [?]}} }}
	\newcommand{\proposal}[1]{{\textcolor{blue}{\em #1}\marginpar{{\bf [\&]}} }}
	\newcommand{\eigenname}[1]	{{\em #1}}

	%Englisch	
		\newcommand{\figref}[1]		{Figure~\ref{fig:#1}}
		\newcommand{\tabref}[1]		{Table~\ref{tab:#1}}
		\newcommand{\equref}[1]		{Equation~\ref{equ:#1}}
		\newcommand{\algref}[1]		{Algorithm~\ref{alg:#1}}
		\newcommand{\defref}[1]		{Definition~\ref{def:#1}}
		\newcommand{\quoteref}[1]	{Quote~\ref{quote:#1}}
		
		\newcommand{\chapref}[1]	{Chapter~\ref{cha:#1}}
		\newcommand{\appref}[1]		{Appendix~\ref{cha:#1}}
		\newcommand{\secref}[1]		{Section~\ref{sec:#1}}
		\newcommand{\ssecref}[1]	{Section~\ref{ssec:#1}}
		\newcommand{\sssecref}[1]	{Section~\ref{sssec:#1}}
		
	% REDEFINE UGLY STUFF
		\renewcommand{\leq}				{\leqslant}
		\renewcommand{\geq}       {\geqslant}
		\renewcommand{\epsilon}   {\varepsilon}
		\newcommand{\musec}       {$\mu sec$\xspace}
		\newcommand{\muW}       	{$\mu W$\xspace}
		\newcommand{\plusminus}		{$\pm $\xspace}
	
%------------------------------------------------------------------------------
%- Worttrennung
%------------------------------------------------------------------------------
	
	%\hyphenation{Ge-samt-ozon}	
	\hyphenation{name-space}	
	\hyphenation{name-spaces}	

			
%------------------------------------------------------------------------------
%- Grafiken
%------------------------------------------------------------------------------
	
	% Vereinfacht die Einbettung von Grafiken
	% Beispiel: \myfig[5cm]{psdatei}{�bersicht �ber das Gesamtsystem}
	\newcommand{\myfig}[3][\columnwidth]
	{
	 \begin{figure}[htbp]
		 \begin{center}
			 \includegraphics[width=#1]{img/#2}
			 \caption{#3}
			 \label{fig:#2}
		 \end{center}
	 \end{figure}
	}
	
	\newcommand{\myfigtwo}[4][\columnwidth]
	{
		 \begin{figure}[htbp]
				\begin{center}
				  \mbox
				  {
				    \subfigure[#2] 
				    { \includegraphics[width=.45\columnwidth]{img/#1-a} \label{fig:#1-a} } 
				    \quad
				    \subfigure[#3]
				    { \includegraphics[width=.45\columnwidth]{img/#1-b} \label{fig:#1-b} }
			    }
				  \caption{#4}
					\label{fig:#1}
				\end{center}
			\end{figure}
	}
	
	\newcommand{\myfigthree}[5][\columnwidth]
	{
		 \begin{figure}[htbp]
				\begin{center}
				  \mbox{
				    \subfigure[#2]
				    {
				    	\includegraphics[width=.3\columnwidth]{img/#1-a}
				    	\label{fig:#1-a}
				    } 
%				    \quad
				    \subfigure[#3]
				    {
							\includegraphics[width=.3\columnwidth]{img/#1-b}
				    	\label{fig:#1-b}
				    }
%				    \quad
				    \subfigure[#4]
				    {
							\includegraphics[width=.3\columnwidth]{img/#1-c}
				    	\label{fig:#1-c}
				    }
			    }	
				  \caption{#5}
					\label{fig:#1}
				\end{center}
			\end{figure}
	}
	
	\newcommand{\myfigfour}[6][\columnwidth]
	{
		 \begin{figure}[htbp]
				\begin{center}
				  \mbox
				  {
				    \subfigure[#2] 
				    { \includegraphics[width=.45\columnwidth]{img/#1-a} \label{fig:#1-a} } 
				    \quad
				    \subfigure[#3]
				    { \includegraphics[width=.45\columnwidth]{img/#1-b} \label{fig:#1-b} }
			    }
				  \mbox
				  {
				    \subfigure[#4] 
				    { \includegraphics[width=.45\columnwidth]{img/#1-c} \label{fig:#1-c} } 
				    \quad
				    \subfigure[#5]
				    { \includegraphics[width=.45\columnwidth]{img/#1-d} \label{fig:#1-d} }
			    }
			    
				  \caption{#6}
				\label{fig:#1}
				\end{center}
			\end{figure}
	}
	
	

\newenvironment{apidoc}[6]%
{%

		\begin{longtable}[t]{p{2cm}p{12.8cm}}
      Semantics	 & #2 \\[0.5em]
	    Signature	 & \lstinline{#1}\\[0.5em]
	    Parameters & {#3} \\[0.5em]
      Returns		 & #4 \\[0.5em]
      Rationale \tabularnewline for changes & #5 \\[0.5em]
      Since 		 & #6 \\[0.5em]
		\end{longtable}

 	
%	}
	\begingroup
%  }%
%  {
  \endgroup
}	


\newcommand{\apiparameters}[1]	{\hspace{-0.25cm}\begin{tabular}[t]{lp{10cm}} #1 \end{tabular}}

\newcommand{\apiparam}[2]	{ \lstinline{#1} & #2 \\ }

